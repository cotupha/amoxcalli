%%%%%%%%%%%%%%%%%%%%%%%%%%%%%%%%%%%%%%%%%%%%%%%%%%%%%%%%%%%%%%%%%%%%%%%%%%%%%%%%%%%%%%%%%%%%%%%%
%%% This is a sample file that shows how to use the commands defined in amoxcally.sty in order %
%%% to produce multiple bibliographies within one file, restarting and keeping the numbering   %
%%% according to the user's needs, as well as using counters in order to know the total number %
%%% of bibliographic entries displayed.                                                        %
%%% Although the documentation is written in English, it was initially intented for use with   %
%%% documents in Spanish (as you can see in the amoxcalliStyle.bst                             %
%%%%%%%%%%%%%%%%%%%%%%%%%%%%%%%%%%%%%%%%%%%%%%%%%%%%%%%%%%%%%%%%%%%%%%%%%%%%%%%%%%%%%%%%%%%%%%%%


\documentclass[11pt,spanish]{article}

\usepackage{amoxcalli}

\title{Sample file for \textsc{amoxcalli}}
\author{cotupha}
\sloppy

%%%COUNTERS
%%%We define the counters we will use for the example.

%%%Counters for the books
\newtotcounter{totalCitasAbductiveReasoning}
\newtotcounter{totalCitasSeekingExplanations}
\newtotcounter{totalCitasLibros}
%%%Command to calculate the total number of of entries corresponding to books.
\newcommand{\saveTotalLibros}{%
\addtocounter{totalCitasLibros}{\totvalue{totalCitasAbductiveReasoning}}
\addtocounter{totalCitasLibros}{\totvalue{totalCitasSeekingExplanations}}
}

%%%Counters for the articles
\newtotcounter{totalCitasLogicsScientificDiscovery}
\newtotcounter{totalCitasAbduccionCambioEpistemico}
\newtotcounter{totalCitasArticulos}

%%%Command to calculate the total number of of entries corresponding to articles.
\newcommand{\saveTotalArticulos}{%
\addtocounter{totalCitasArticulos}{\totvalue{totalCitasLogicsScientificDiscovery}}
\addtocounter{totalCitasArticulos}{\totvalue{totalCitasAbduccionCambioEpistemico}}
}

%%%Counter for ALL the entries.
\newtotcounter{totalCitasGeneral}
%%%Command to calculate the total number of entries (book entries + article entries)
\newcommand{\saveTotalGeneral}{%
\addtocounter{totalCitasGeneral}{\totvalue{totalCitasLibros}}
\addtocounter{totalCitasGeneral}{\totvalue{totalCitasArticulos}}
}

\bibliographystyle{amoxcalliStyle}

\begin{document}

\selectlanguage{spanish}

\maketitle

\section*{Summary}

This a sample reduced version of papers citing two books and two articles, whose details are given below. 
Many publications and years have been removed from the original document.

\textsc{\textbf{Total number of bibliography entries: \total{totalCitasGeneral}.}}

\textsc{\textbf{Total number of bibliography entries citing the books: \total{totalCitasLibros}.}}

\textsc{\textbf{Total number of bibliography entries citing the articles: \total{totalCitasArticulos}.}}

\section*{Books}
\subsection*{\textsc{Atocha Aliseda}.
\newblock \emph{{Abductive Reasoning. Logical investigations into Discovery and
  Explanation}}, tomo 330.
\newblock Synthese Library. Springer, 2006.
\newblock ISBN 978-1-4020-3906-5.
}

Total number of entries for this section: \printPubNum{totalCitasAbductiveReasoning}

\restartNumb

\begin{btUnit}
 \begin{btSect}{AbductionPapers}
  \section*{2014}
  \nocite{Magnani:UnderstandingAbduction}
  \nocite{Woods:AgainstFictionalism}
  \btPrintCited
 \end{btSect}
\end{btUnit}

\keepNumb

\begin{btUnit}
 \begin{btSect}{AbductionPapers}
  \section*{2013}
  \nocite{CordonDitmarschNepomuceno:DynamicConsequence}
  \nocite{Magnani:AbductionIgnorancePreserving}
  \nocite{PauwelsMeyerCampenhout:DesignThinkingSupport}
  \btPrintCited
 \end{btSect}
\end{btUnit}

\begin{btUnit}
 \begin{btSect}{AbductionPapers}
   \section*{2012}
    \nocite{NepomucenoSolerVelazquez:DinamicaInformacionAgentes}
    \nocite{Woods:CognitiveEconomicsAbduction}
    \btPrintCited
  \end{btSect}   
 \end{btUnit}

\saveNumb{totalCitasAbductiveReasoning}
\subsection*{\textsc{Atocha Aliseda}.
\newblock \emph{Seeking Explanations: Abduction in Logic, Philosophy of Science
  and Artificial Intelligence}.
\newblock Institute for Logic, Language and Computation, Amsterdam, 1997.
\newblock ISBN 90-74795-73-0.
}

Total number of entries for this section: \printPubNum{totalCitasSeekingExplanations}

\restartNumb

\begin{btUnit}
 \begin{btSect}{AbductionPapers}
  \section*{2014}
  \nocite{Magnani:UnderstandingAbduction}
  \btPrintCited
 \end{btSect}
\end{btUnit}

\keepNumb

\begin{btUnit}
 \begin{btSect}{AbductionPapers}
  \section*{2013}
  \nocite{Mackonis:InferenceBestExplanation}
  \nocite{Magnani:AbductionIgnorancePreserving}
  \btPrintCited
 \end{btSect}
\end{btUnit}


\begin{btUnit}
 \begin{btSect}{AbductionPapers}
  \section*{2012}
    \nocite{Finger:AutomatedFOAbduction}
    \nocite{FranconiNgoSherkhonov:DefinabilityAbductionProblem}
    \nocite{Rast:NonindexicalContext}
  \btPrintCited
 \end{btSect}
\end{btUnit}

\saveNumb{totalCitasSeekingExplanations}
\saveTotalLibros

\section*{Articles}
\subsection*{\textsc{Atocha Aliseda}.
\newblock \menquote{Logics in Scientific Discovery}.
\newblock \emph{Foundations of Science}, 9(3), p\'ags. 339--363, 2004.
\newblock ISSN 1233-1821.
}
Total number of entries for this section: \printPubNum{totalCitasAbductiveReasoning}

\restartNumb

\begin{btUnit}
 \begin{btSect}{AbductionPapers}
  \section*{2013}
  \nocite{PauwelsMeyerCampenhout:DesignThinkingSupport}
  \nocite{TohmeCrespo:AbductionEconomics}
  \btPrintCited
 \end{btSect}
\end{btUnit}

\keepNumb

\begin{btUnit}
 \begin{btSect}{AbductionPapers}
 \section*{2011}
 \nocite{Ramirez:InfAbdModelos}
 \btPrintCited
 \end{btSect}
\end{btUnit}

\saveNumb{totalCitasLogicsScientificDiscovery}
\subsection*{\textsc{Atocha Aliseda}.
\newblock \menquote{{La abducci{\'o}n como cambio epist{\'e}mico: CS Peirce y
  las teor{\'i}as epist{\'e}micas en inteligencia artificial}}.
\newblock \emph{Analog{\'i}a Filos{\'o}fica,\menquote{Charles S. Peirce y la
  Abducci{\'o}n}}, 12(1), p\'ags. 125--144, 1998.
\newblock ISSN 0188-896X.
}
Total number of entries for this section: \printPubNum{totalCitasAbductiveReasoning}

\restartNumb

\begin{btUnit}
 \begin{btSect}{AbductionPapers}
  \section*{2013}
  \nocite{Ramirez:PosibilidadJustificacionDeduccion}
  \nocite{Nunez:EpistemologieDecouverte}
  \btPrintCited
 \end{btSect}
\end{btUnit}

\keepNumb

\begin{btUnit}
 \begin{btSect}{AbductionPapers}
  \section*{2012}
  \nocite{Jesper:BrilloImagen}
  \nocite{Nubiola:NewDevelopmentsPeirce}
  \btPrintCited
 \end{btSect}
\end{btUnit}

\saveNumb{totalCitasAbduccionCambioEpistemico}
\saveTotalArticulos

\saveTotalGeneral

\end{document}